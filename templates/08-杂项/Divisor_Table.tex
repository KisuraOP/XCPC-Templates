$\omega(n)$:不同质因子的个数,如 $\omega(12)=\omega(2^2\times 3) = 2$。

$d(n)$:正因数个数,如 $d(12)=6$(因数有 $1,2,3,4,6,12$)。


\[
\begin{array}{|c|c|c|c|c|c|c|c|c|c|}
\hline
n \leq & 10^1 & 10^2 & 10^3 & 10^4 & 10^5 & 10^6 & 10^7 & 10^8 & 10^9 \\
\hline
\max\{\omega(n)\} & 2 & 3 & 4 & 5 & 6 & 7 & 8 & 8 & 9 \\
\hline
\max\{d(n)\} & 4 & 12 & 32 & 64 & 128 & 240 & 448 & 768 & 1344 \\
\hline
n \leq & 10^{10} & 10^{11} & 10^{12} & 10^{13} & 10^{14} & 10^{15} & 10^{16} & 10^{17} & 10^{18} \\
\hline
\max\{\omega(n)\} & 10 & 10 & 11 & 12 & 12 & 13 & 13 & 14 & 15 \\
\hline
\max\{d(n)\} & 2304 & 4032 & 6720 & 10752 & 17280 & 26880 & 41472 & 64512 & 103680 \\
\hline
\end{array}
\]